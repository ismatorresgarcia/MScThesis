\addchap{Resumen}
\lettrine{D}{esde la aparición de los primeros láseres} a principios de $1960$, estos complejos sistemas ópticos han encontrado multitud de aplicaciones tanto en investigación científica como en la industria tecnológica, transforman diversos campos gracias a sus increíbles propiedades físicas y gran versatilidad. En el ámbito científico, los láseres se utilizan constantemente en espectroscopia, permitiendo analizar la composición de materiales a escala molecular o la estructura de microorganismos como bacterias y virus. 

Sus aportaciones en física y química, que han llegado a reunir más de una decena de Premios Nobel, han conseguido alcanzar hitos extraordinarios, relacionados con disciplinas como la óptica cuántica, la óptica ultrarrápida o la nanotecnología. En la industria, se han convertido en instrumentos indispensables de corte, soldadura y marcado de materiales con alta precisión. En las telecomunicaciones, son cruciales en la transmisión de información a través de fibras ópticas, mejorando la velocidad y la eficacia de las comunicaciones. En medicina, muchos procedimientos quirúrgicos emplean sistemas láser para realizar operaciones, además de estar presentes también en el diagnóstico clínico y tratamiento de enfermedades como diferentes tipos de cáncer.

En este Trabajo Fin de Máster, se estudia la interacción de un armónico de alto orden (\acrshort{hoh}) con un plasma amplificador de radiación ultravioleta extrema o rayos X blandos (\acrshort{xuv}) formado por iones de kriptón altamente ionizados (\ce{Kr^{8+}}), responsables de la amplificación. Mediante la realización de simulaciones numéricas, el objetivo es reproducir las condiciones y resultados obtenidos experimentalmente, ayudando en la comprensión de los distintos fenómenos físicos que participan en el proceso de amplificación del armónico.

Estas simulaciones están centradas en la utilización del código Dagon\autocite{Oliva2017}, encargado de modelizar la amplificación tridimensional del armónico inyectado a través de una columna de plasma, introduciendo las ecuaciones de Maxwell-Bloch para su resolución numérica. El código fue desarrollado en el Instituto de Fusión Nuclear \enquote{Guillermo-Velarde} (IFN-GV) de la Escuela Técnica Superior de Ingenieros Industriales (ETSII), en la Universidad Politécnica de Madrid (UPM), a partir del esfuerzo realizado por el Dr. Eduardo Oliva Gonzalo ---tutor y guía del trabajo--- para estudiar láseres de rayos X blandos basados en plasmas (\acrshort{sxrl}).

Los antecedentes que motivaron la aparición de este trabajo fueron varios experimentos\autocite{Tuitje2020,Depresseux2015} llevados a cabo en el \emph{\acrfull{loa}}, situado en París, donde un gas de kriptón a alta presión es fuertemente ionizado mediante un sistema de varios pulsos láser infrarrojos, formándose un canal de plasma capaz de amplificar un haz de rayos X blandos ($\lambda=\qty{32.8}{nm}$). Estos experimentos buscaban reducir la duración del pulso de radiación ultravioleta producido, además de medir propiedades como la evolución temporal de los fotones o las distribuciones de iones y electrones en el interior del canal de plasma. 

Sin embargo, algunos aspectos relacionados con las curvas de intensidad y fase del haz amplificado no se entendían correctamente al principio. El esquema presentado en este proyecto consiste en modificar la distribución de \ce{Kr^{8+}} en la columna de plasma, introduciendo en Dagon parámetros que permitan controlar la región de abundancia del ión en el interior del medio activo. De esta manera, es posible estudiar sistemáticamente los efectos que tienen cuestiones como la anchura del canal con presencia de \ce{Kr^{8+}} sobre la amplificación, intentando ajustarse a las observaciones experimentales.

A partir de las imágenes obtenidas en las simulaciones ejecutadas, ha sido posible mejorar el entendimiento de la amplificación resultante en las distribuciones de intensidad y fase mencionadas, estudiando la influencia de la geometría de la frontera que alberga el ión \ce{Kr^{8+}} en su interior, variando parámetros como su amplitud en dirección radial a lo largo de la longitud de la columna de plasma.

Especialmente, el postprocesamiento de las imágenes ha revelado la capacidad de replicar perfectamente el efecto de la sobreionización inicial producida durante la formación del canal de plasma, responsable de la formación de un valle en el perfil radial de intensidad del haz a la salida del canal. Además, la tendencia de esta curva también ha podido adaptarse razonablemente a los resultados del laboratorio, así como la profundidad entre máximos y mínimos del perfil radial de fase de los fotones.

A pesar de esto, el acuerdo conseguido entre simulación y experimento todavía necesita mejorar la precisión, haciendo énfasis en la densidad electrónica o en otros aspectos de la concentración de \ce{Kr^{8+}} dentro del canal de plasma, que pudieran reducir las discrepancias observadas. Nuevas investigaciones y proyectos deberán concentrar sus esfuerzos en proponer nuevas modificaciones en la distribución de iones y electrones, partiendo de las observaciones realizadas en trabajos anteriores.

De este modo, en el futuro podrían utilizarse estas fuentes de radiación para la diagnosis y reconstrucción de imágenes en tres dimensiones, sustituyendo tal vez grandes instalaciones ---empleadas, por ejemplo, en los láseres de electrones libres (\acrshort{fel})--- cuya operación resulta más costosa y compleja de mantener. 

En definitiva, este trabajo está integrado dentro de una amplia línea de investigación dedicada a comprender la amplificación de haces de rayos X blandos mediante plasmas densos que interaccionan con pulsos de armónicos de alto orden. Los resultados obtenidos demuestran la posibilidad de reproducir mediante simulaciones numéricas aspectos fundamentales de la intensidad y fase del pulso ultravioleta extremo a la salida del medio activo. 

\begin{table}[htpb]
    \begin{tabular}{l}
        \textbf{Palabras Clave} \\
        \midrule
         Canal de plasma, láser XUV, armónicos de alto orden, código Dagon, Maxwell-Bloch. 
    \end{tabular}
\end{table}

\begin{table}[htpb]
    \begin{tabular}{ll}
        \textbf{Códigos UNESCO} & \\
        \midrule
         220910 & Láseres \\
         220909 & Radiación infrarroja \\
         220212 & Rayos X \\
         220702 & Iones atómicos \\
         220410 & Física de plasmas \\
         220407 & Ionización 
    \end{tabular}
\end{table}
