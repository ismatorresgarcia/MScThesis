% !TeX program = lualatex
% Energy levels used in the nickel-like quasi-3-level scheme of an eight-folded kripton seeded laser-plasma amplifier
% Autor: Ismael Torres García
\documentclass[tikz,border=3pt]{standalone}
\usetikzlibrary{calc, arrows.meta, decorations.pathmorphing, decorations.markings}
\usepackage[math-style=ISO,bold-style=ISO]{unicode-math}
\setmainfont[Renderer=OpenType]{Minion Pro}          %% Fuente comercial MinionPro de Adobe para el texto
\setmathfont[                                        %% Fuente MinionMath de Johannes Küsner para las matemáticas
    Extension = .otf,
    Path = /Users/ytoga/Library/Fonts/,
    Scale = 1,
    Script = Math,
    SizeFeatures = {
        {Size = -6, Font = MinionMath-Tiny,
        Style = MathScriptScript},
        {Size = 6-8.4, Font = MinionMath-Capt,
        Style = MathScript},
        {Size = 8.4-, Font = MinionMath-Regular},
    },
]{MinionMath-Regular}
\setmathfont[                         
    Extension = .otf,
    Path = /Users/ytoga/Library/Fonts/,
    version = bold,
]{MinionMath-Bold}

\usepackage{xcolor}
\definecolor{laser}{RGB}{76,0,155}    % Color de la semilla

%%%%%%%%%%%%%%%%%%%%%%%%%%
%%% COMIENZO DEL DOCUMENTO
%%%%%%%%%%%%%%%%%%%%%%%%%%
\begin{document}

\begin{tikzpicture}[
  arrow/.style={-{Stealth[length=2mm,width=1.5mm]},shorten >= 1pt},
  bombeo/.style={arrow,thick},
  sfestim/.style={thick,decorate,decoration={snake,amplitude=2.0}},
  estim/.style={arrow,thick,decorate,decoration={snake,amplitude=2.0,post length=5}},
  espont/.style={arrow,densely dashed,decorate,decoration={snake,amplitude=2.0,post length=5}},
  semilla/.style={arrow,laser,thick,decorate,decoration={snake,amplitude=6,segment length=8,post length=7}},
]

%% DEFINICIONES
\def\anchura{90pt}
\coordinate (sublevel) at (0, 10pt);
\coordinate (sublevel2) at (125pt, 155pt);
\coordinate (sublevel3) at (250pt, 55pt);

%% NIVELES FUNDAMENTAL Y SUPERIOR
\coordinate (S00) at ($-1*(sublevel)$);
\coordinate (S01) at ($23*(sublevel)$);

% Dibujo los niveles de energía 
\draw[ultra thick] (S00) -- +(340pt, 0) node[midway, below=5pt] {\footnotesize $\mathrm{3d^{10}} (J=0)$};
\draw[thick] (S01) -- +(\anchura, 0) node[anchor=south] {\footnotesize $\mathrm{3d^94f}$};

%% NIVELES SUPERIORES DE LA TRANSICIÓN LASER
\coordinate (T00) at (sublevel2);
\coordinate (T01) at ($(T00) - 2*(sublevel)$);
\coordinate (T02) at ($(T01) - 2*(sublevel)$);

% Dibujo los niveles de energía
\draw[thick] (T00) node[left=0pt] {\scriptsize $(\mathrm{3d_{3/2},4d_{3/2}}) J=0$} -- +(\anchura, 0) 
node[anchor=south] {\footnotesize $\mathrm{3d^94d}$};
\draw[thick] (T01) node[left=0pt] {\scriptsize $(\mathrm{3d_{3/2},4d_{3/2}}) J=2$} -- +(\anchura, 0);
\draw[thick] (T02) node[left=0pt] {\scriptsize $(\mathrm{3d_{3/2},4d_{3/2}}) J=1$} -- +(\anchura, 0);

% Dibujo las transiciones del bombeo
\draw[arrow] ([xshift=20pt] S00) -- ([xshift=20pt] S01);
\draw[dashed] ([xshift=80pt] S01) -- ([xshift=10pt] T00);
\draw[arrow, dashed] ([xshift=10pt] T00) -- ([xshift=10pt] T02);

\draw[bombeo] ([xshift=160pt] S00) -- ([xshift=35pt] T00);
\draw[bombeo] ([xshift=170pt] S00) -- ([xshift=45pt] T01);
\draw[bombeo] ([xshift=180pt] S00) -- ([xshift=55pt] T02);

%% NIVELES INFERIORES DE LA TRANSICIÓN LASER
\coordinate (U00) at (sublevel3);
\coordinate (U01) at ($(U00) - 2*(sublevel)$);
\coordinate (Si) at (290pt,145pt);
\coordinate (Sf) at (255pt,120pt);

% Dibujo los niveles de energía
\draw[thick] (U00) node[left=0pt] {\scriptsize $(\mathrm{3d_{5/2},4p_{3/2}}) J=1$} -- +(\anchura,0) node[anchor=south] {\footnotesize $\mathrm{3d^94p}$};
\draw[thick] (U01) node[left=0pt] {\scriptsize $(\mathrm{3d_{3/2},4p_{1/2}}) J=1$} -- +(\anchura,0);

% Dibujo las transiciones espontáneas
\draw[espont] ([xshift=80pt] U00) -- ([xshift=330pt] S00);
\draw[espont] ([xshift=70pt] U01) -- ([xshift=320pt] S00);

% Dibujo las transiciones estimuladas
\draw[sfestim] ([xshift=90pt] T00) -- ([xshift=55pt] U00);
\draw[estim] ([xshift=55pt] U00) -- ([xshift=55pt] U01);
\draw[estim] ([xshift=75pt] T00) -- ([xshift=40pt] U00);
\draw[estim] ([xshift=75pt] T01) -- ([xshift=25pt] U00);
\draw[estim] ([xshift=75pt] T02) -- ([xshift=10pt] U00);

% Dibujo la semilla
\draw[semilla] (Si) node[above, align=center] {\footnotesize HHG\\\footnotesize (Armónico $25$)} -- (Sf);

%% TEXTO
\node[left, align=center] at (110pt,190pt) {\footnotesize Emisión\\\footnotesize no radiativa};
\node[left, align=center] at (160pt,60pt) {\footnotesize Excitación\\\footnotesize colisional\\\footnotesize (Bombeo)};
\node[right, fill=white, align=center] at (210pt,91pt){\footnotesize Emisión estimulada\\\footnotesize (XUV)};
\node[right, align=center] at (270pt,15pt) {\footnotesize Emisión\\\footnotesize espontánea};

\end{tikzpicture}

\end{document}

%\begin{figure}
 %   \centering
  %  \includestandalone[width=\textwidth]{Tikz/laser_tran}
   % \caption{Esquema de cuasi-3-niveles empleado para los iones de Kr$^{8+}$ del canal de plasma. La amplificación de la semilla de armónicos de alto orden (HHG) ocurre para la transición entre los niveles $3d^94d$-$3d^94p$ a \qty{32,8}{\nm}, emitiendo rayos X blandos (XUV) de alta coherencia espacial y temporal.}
   % \label{fig:niveles}
%\end{figure}
