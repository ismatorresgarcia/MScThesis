% !TeX program = lualatex
% Stakeholders and impacts related with the Master's Thesis
% Autor: Ismael Torres García
\documentclass[tikz,border=3pt]{standalone}
\usetikzlibrary{arrows.meta}
\usepackage[math-style=ISO,bold-style=ISO]{unicode-math}
\setmainfont[Renderer=OpenType]{Minion Pro}          %% Fuente comercial MinionPro de Adobe para el texto
\setmathfont[                                        %% Fuente MinionMath de Johannes Küsner para las matemáticas
    Extension = .otf,
    Path = /Users/ytoga/Library/Fonts/,
    Scale = 1,
    Script = Math,
    SizeFeatures = {
        {Size = -6, Font = MinionMath-Tiny,
        Style = MathScriptScript},
        {Size = 6-8.4, Font = MinionMath-Capt,
        Style = MathScript},
        {Size = 8.4-, Font = MinionMath-Regular},
    },
]{MinionMath-Regular}
\setmathfont[                         
    Extension = .otf,
    Path = /Users/ytoga/Library/Fonts/,
    version = bold,
]{MinionMath-Bold}
\setmathfont{STIXTwoMath}[          %% Fuente STIX Two Math para los estilos matemáticos que faltan en MinionMath
    Extension = .otf,
    Path = /System/Library/Fonts/Supplemental/,
    range = {\symcal,\symfrak,\symbfcal,\symbffrak},
    StylisticSet = 1
]

\usepackage{xcolor}
%%%%%%%%%%%%%%%%
% DEFINICIONES %
%%%%%%%%%%%%%%%%
\def\circRad{4em}
\definecolor{miazul}{RGB}{0,30,155} 
\definecolor{mimorado}{RGB}{140,45,165}
\definecolor{minaranja}{RGB}{250,150,5}
\definecolor{mirojo}{RGB}{180,0,30} 
\definecolor{miverde}{RGB}{0,153,0}

\tikzset{
  line cap = round, thick,
  circulo/.style = {shape=circle, draw, font=\bfseries, minimum width=2*\circRad},
  etiqueta/.style = {draw, thin, inner sep=2, rounded corners=2},
  every node/.style = {align=center}
}

\begin{document}

\begin{tikzpicture}

\begin{scope}[local bounding box=challenges]
  % Ambiental
  \node [circulo, fill=minaranja!30] (amb) {Impacto\\Ambiental};
  \foreach \itm [count=\i, evaluate={\a=\i*15+120;}] in
    {Equipos, Ordenador personal, Servidor remoto, Consumo eléctrico, Conexión a red, Fabricación, Materias primas, Transporte} {
      \node[etiqueta] at (\a:\circRad + 2mm) [rotate=\a+180, anchor=east] {\footnotesize\itm};
      \draw (\a:\circRad + 2mm) -- (\a:\circRad);
    }

  % Económico
  \begin{scope}[xshift=6cm]
    \node [circulo, fill=mimorado!30] (eco) {Impacto\\Económico};
    \foreach \itm [count=\i, evaluate={\a=\i*15+140;}] in
      {Reducción de costes, Inversión, Mantenimiento, Desmantelación, Industria} {
        \node[etiqueta] at (\a:\circRad + 2mm) [rotate=\a+180, anchor=east] {\footnotesize\itm};
        \draw (\a:\circRad + 2mm) -- (\a:\circRad);
      }
    \foreach \itm [count=\i, evaluate={\a=30-\i*15;}] in
      {Mayor eficiencia energética, Más empleos, Más financiación} {
      \node[etiqueta] at (\a:\circRad + 2mm) [rotate=\a, anchor=west] {\footnotesize\itm};
      \draw (\a:\circRad + 2mm) -- (\a:\circRad);
      }
  \end{scope}

  % Social
  \begin{scope}[xshift=12cm]
    \node [circulo, fill=miazul!30] (soc) {Impacto\\Social};
    \foreach \itm [count=\i, evaluate={\a=70-\i*16;}] in
      {Biología molecular, Bioquímica, Medicina, Cirugía, Salud, Bienestar} {
        \node[etiqueta] at (\a:\circRad + 2mm) [rotate=\a, anchor=west] {\footnotesize\itm};
        \draw (\a:\circRad + 2mm) -- (\a:\circRad);
      }
  \end{scope}

  % Grupos de interés directos
  \begin{scope}[xshift=3cm,yshift=-6cm]
    \node [circulo, fill=miverde!30] (gdir) {Grupos\\Directos};
    \foreach \itm [count=\i, evaluate={\a=15*\i+120;}] in
      {Ismael Torres García, Eduardo Oliva Gonzalo, Instituto de Fusión Nuclear, Departamento de Ingeniería Energética, Escuela Técnica Superior de Ingenieros Industriales, Universidad Politécnica de Madrid} {
        \node[etiqueta] at (\a:\circRad + 2mm) [rotate=\a+180, anchor=east] {\footnotesize\itm};
        \draw (\a:\circRad + 2mm) -- (\a:\circRad);
      }
  \end{scope}

  % Grupos de interés indirectos
  \begin{scope}[xshift=9cm,yshift=-6cm]
    \node [circulo, fill=mirojo!30] (gir) {Grupos\\Indirectos};
    \foreach \itm [count=\i, evaluate={\a=70-\i*16;}] in
      {Comunidad de Madrid, Ministerio de Ciencia, Unión Europea, Laboratoire d'Optique Apliquée, École Polytechnique} {
        \node[etiqueta] at (\a:\circRad + 2mm) [rotate=\a, anchor=west] {\footnotesize\itm};
        \draw (\a:\circRad + 2mm) -- (\a:\circRad);
      }
  \end{scope}
\end{scope}

\draw[-] (amb.-70) to [bend right] (eco.-110);
\draw[-] (eco.-70) to [bend right] (soc.-110);
\draw[-] (amb.-70) to [bend left] (gdir.110);
\draw[-] (gdir.70) to [bend left] (gir.110);
\draw[-] (gir.70) to [bend left] (soc.-110);
\draw[-] (gir.70) to [bend left] (soc.-110);

\end{tikzpicture}

\end{document}
