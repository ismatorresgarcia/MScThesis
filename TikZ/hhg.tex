% !TeX program = lualatex
% High order harmonics generation through a cell filled with argon gas
% Autor: Ismael Torres García
\documentclass[tikz,border=3pt]{standalone}
\usetikzlibrary{calc, arrows.meta, fadings}
\usepackage[math-style=ISO,bold-style=ISO]{unicode-math}
\setmainfont[Renderer=OpenType]{Minion Pro}          %% Fuente comercial MinionPro de Adobe para el texto
\setmathfont[                                        %% Fuente MinionMath de Johannes Küsner para las matemáticas
    Extension = .otf,
    Path = /Users/ytoga/Library/Fonts/,
    Scale = 1,
    Script = Math,
    SizeFeatures = {
        {Size = -6, Font = MinionMath-Tiny,
        Style = MathScriptScript},
        {Size = 6-8.4, Font = MinionMath-Capt,
        Style = MathScript},
        {Size = 8.4-, Font = MinionMath-Regular},
    },
]{MinionMath-Regular}
\setmathfont[                         
    Extension = .otf,
    Path = /Users/ytoga/Library/Fonts/,
    version = bold,
]{MinionMath-Bold}

\usepackage{siunitx}
\sisetup{
  detect-all, 
  output-decimal-marker={\text{,}},
}
\usepackage{xcolor}
\definecolor{kr}{RGB}{51,153,255}       % Color del kriptón
\definecolor{ar}{RGB}{255,102,255}     % Color de los átomos
\definecolor{elc}{RGB}{204,204,0}       % Color de los electrones
\definecolor{laseruv}{RGB}{76,0,155}   % Color del armónico
\definecolor{laserir}{RGB}{255,51,51}  % Color del IR
\definecolor{celda}{RGB}{240,240,240}  % Color de la celda de cristal
\definecolor{filtro}{RGB}{132,135,137}  % Color del filtro de aluminio

%%%%%%%%%%%%%%%%%%%%%%%%%%
%%% COMIENZO DEL DOCUMENTO
%%%%%%%%%%%%%%%%%%%%%%%%%%
\begin{document}

%% DECORACIÓN DE LOS ÁTOMOS DE ARGÓN, EL LÁSER Y LA CELDA
\tikzset{  
  argon/.pic={
  \tikzset{/argon/.cd,#1}
  \node[circle, fill, inner sep=1.2, ball color=ar] at 
  (0,0) {};
  }
  /argon/.search also={/tikz},
  /argon/.cd,
}
\tikzset{
  kripton/.pic={
  \tikzset{/kripton/.cd,#1}
  \node[circle,fill,inner sep=1.2,ball color=kr] at (0,0) {};
  }
  /kripton/.search also={/tikz},
  /kripton/.cd,
}
\tikzset{  
  elec/.pic={
  \tikzset{/elec/.cd,#1}
  \node[circle,fill,inner sep=0.8,ball color=elc] at (0,0) {};
  }
  /elec/.search also={/tikz},
  /elec/.cd,
}

\begin{tikzpicture}

%%%%%%%%%%%%%%%%
%%% DIBUJO 1 %%%
%%%%%%%%%%%%%%%%

%%% DEFINICIONES
%\def\r{7}                  % Radio de los haces
%\def\divir{12}             % Divergencia del IR
%\def\divuv{6}              % Divergencia del UV
%\def\largo{1.5}            % Mitad del largo de la celda
%\def\ancho{0.6}            % Mitad del ancho de la celda
%\def\cols{15}              % Número de columnas
%\def\rows{6}               % Número de filas
%\def\lam{0.3}              % Ancho del filtro de Al
%\def\SquareUnit{0.2}       % Lado de la celda pequeña
%
%%% COORDENADAS
%\coordinate (O) at (0:0);
%\coordinate (A) at (-\divir:-\r);
%\coordinate (B) at (\divir:-\r);
%\coordinate (C) at (\divuv:\r);
%\coordinate (D) at (-\divuv:\r);
%\coordinate (E1) at (-\largo,-\ancho);
%\coordinate (E2) at (\largo,\ancho);
%
%%% GENERACIÓN DEL ARMÓNICO DE ALTO ORDEN
%\fill[laserir, path fading=west] (O) -- (A) -- (B) -- cycle;
%\fill[laseruv, path fading=east] (O) -- (C) -- (D) -- cycle;
%
%%% ARGÓN
%\begin{scope}[fill opacity=0.5]
%  \filldraw[fill=celda, draw=gray!30] (E1) rectangle (E2);
%  \foreach \i in {1,...,\cols}{
%	  \foreach \j in {1,...,\rows}{
%      \pgfmathsetmacro\l{\SquareUnit-0.001*rnd}
%      \pgfmathsetmacro\x{(\i-1)*\SquareUnit+\l*rnd}
%      \pgfmathsetmacro\y{(\j-1)*\SquareUnit+\l*rnd}
%      \pic at ([shift=(E1)] \x,\y) {argon};
%    }
%  }
%\end{scope}
%
%%% TEXTO
%\node[anchor=south west, rotate=-\divir] at (A) {\footnotesize Pulso NIR (\qty{30}{\mJ}, \qty{30}{\fs})};
%\node[anchor=south east, rotate=\divuv] at (C) {\footnotesize Armónico de alto orden};
%\node[anchor=north] at ($(O)+(0,-\ancho)$) {\footnotesize Celda con gas argón};
    
%%%%%%%%%%%%%%%%
%%% DIBUJO 2 %%%
%%%%%%%%%%%%%%%%

%% DEFINICIONES
\def\r{7}                             % Radio de los haces
\def\divir{12}                        % Divergencia del IR
\def\divuv{6}                         % Divergencia del UV
\def\largo{1.5}                       % Mitad del largo del canal
\def\ancho{0.6}                       % Mitad del ancho del canal
\def\lam{0.3}                         % Ancho del filtro de Al
\def\cols{10}                         % Número de columnas
\def\rows{4}                          % Número de filas
\def\SquareUnit{0.3}                  % Lado de la celda pequeña
\pgfmathsetmacro\Paraux{0.12}         % Parámetro auxiliar

%% COORDENADAS
\coordinate (O) at (0:0);
\coordinate (A) at (\divir:0.5*\r);
\coordinate (B) at (-\divir:0.5*\r);
\coordinate (C) at (-\divir:-\r);
\coordinate (D) at (\divir:-\r);
\coordinate (E) at (\divuv:\r);
\coordinate (F) at (-\divuv:\r);
\coordinate (G) at (\divuv:-\r);
\coordinate (H) at (-\divuv:-\r);
\coordinate (E1) at (-\largo,-\ancho);
\coordinate (E2) at ($(E1)+(0.1,0.1)$);
\coordinate (E3) at (0.5*\r-\lam,-1);
\coordinate (E4) at (0.5*\r,1);

%% GENERACIÓN DEL SXRL
\fill[laserir, path fading=east] (O) -- (A) -- (B) -- cycle;
\fill[laserir, path fading=west] (O) -- (C) -- (D) -- cycle;
\fill[laseruv, path fading=east] (O) -- (E) -- (F) -- cycle;
\fill[laseruv, path fading=west] (O) -- (G) -- (H) -- cycle;

%% CHORRO DE PLASMA
\foreach \i in {1,...,\cols}{
  \foreach \j in {1,...,\rows}{
    \pgfmathsetmacro\radius{\Paraux*rnd}
    \pgfmathsetmacro\l{\SquareUnit-2*\radius}
    \pgfmathsetmacro\xk{(\i-1)*\SquareUnit+\radius+\l*rnd}
    \pgfmathsetmacro\yk{(\j-1)*\SquareUnit+\radius+\l*rnd}
    \pgfmathsetmacro\xe{(\i-1)*\SquareUnit+\radius+\l*rnd}
    \pgfmathsetmacro\ye{(\j-1)*\SquareUnit+\radius+\l*rnd}
    \pic at ([shift=(E1)] \xk,\yk) {kripton};
    \pic at ([shift=(E2)] \xe,\ye) {elec};
  }
}

%% FILTRO DE ALUMINIO
\filldraw[filtro, fill opacity=0.8] (E3) rectangle (E4);
    
%% TEXTO
\node[anchor=south east, rotate=\divuv] at (E) {\footnotesize Emisión SXRL ó XUV};
\node[anchor=south west, rotate=-\divuv] at (H) {\footnotesize Armónico de alto orden};
\node[anchor=south west, rotate=-\divir] at (C) {\footnotesize Bombeo NIR (\qty{1,6}{\J}, \qty{30}{\fs})};
\node[anchor=north] at ($(O) + (0,-\ancho)$) {\footnotesize Canal de plasma};
\node[anchor=north, align=center] at ($0.5*(E3)+0.5*(E4)+(0,-1)$) {\footnotesize Filtro de \\\footnotesize aluminio};
\end{tikzpicture}

\end{document}
