% !TeX program = lualatex
% Flux diagram used for solving the Maxwell equations in the FBPIC code
% Autor: Ismael Torres García
\documentclass[tikz,border=3pt]{standalone}
\usetikzlibrary{arrows.meta,shadows,shapes.geometric}
\usepackage[math-style=ISO,bold-style=ISO]{unicode-math}
\setmainfont[Renderer=OpenType]{Minion Pro}          %% Fuente comercial MinionPro de Adobe para el texto
\setmathfont[                                        %% Fuente MinionMath de Johannes Küsner para las matemáticas
    Extension = .otf,
    Path = /Users/ytoga/Library/Fonts/,
    Scale = 1,
    Script = Math,
    SizeFeatures = {
        {Size = -6, Font = MinionMath-Tiny,
        Style = MathScriptScript},
        {Size = 6-8.4, Font = MinionMath-Capt,
        Style = MathScript},
        {Size = 8.4-, Font = MinionMath-Regular},
    },
]{MinionMath-Regular}
\setmathfont[                         
    Extension = .otf,
    Path = /Users/ytoga/Library/Fonts/,
    version = bold,
]{MinionMath-Bold}
\setmathfont{STIXTwoMath}[          %% Fuente STIX Two Math para los estilos matemáticos que faltan en MinionMath
    Extension = .otf,
    Path = /System/Library/Fonts/Supplemental/,
    range = {\symcal,\symfrak,\symbfcal,\symbffrak},
    StylisticSet = 1
]

\usepackage{xcolor}
\definecolor{miazul}{RGB}{0,30,155} 
\definecolor{miazull}{RGB}{0,30,180} 
\definecolor{mirojo}{RGB}{180,0,30} 
\definecolor{miverde}{RGB}{0,153,0}
\definecolor{mimorado}{RGB}{140,45,165}
\definecolor{minaranja}{RGB}{250,150,5}

\tikzset{inicio/.style={rectangle, rounded corners, 
      drop shadow,
      minimum width=2cm, 
      minimum height=1cm,
      align=center, 
      draw=black, 
      fill=mirojo!30},
  fin/.style = {rectangle, rounded corners, 
      drop shadow,
      minimum width=2cm, 
      minimum height=1cm,
      align=center, 
      draw=black, 
      fill=miverde!30},
  calculo/.style = {trapezium, drop shadow,
      trapezium stretches=true, 
      trapezium left angle=110, 
      trapezium right angle=70, 
      minimum width=3cm, 
      minimum height=1cm, 
      align=center,
      draw=black, 
      fill=miazul!20},
  proceso/.style = {rectangle, drop shadow,
      minimum width=3cm, 
      minimum height=1cm, 
      align=center, 
      draw=black, 
      fill=minaranja!30},
  decision/.style = {diamond, drop shadow, 
      minimum width=5cm, 
      minimum height=1cm, 
      draw=black, 
      align=center,
      fill=mimorado!30},
  arrow/.style = {-{Stealth[length=2mm,width=1.5mm]},shorten >=1pt}
}

\begin{document}

\begin{tikzpicture}[node distance=3cm]

\node (inicio) [inicio] {Inicio};
\node (push) [proceso, below of=inicio] {Actualización de las magnitudes mecánicas \\ $\symbf{x}_{i}$, $\symbf{p}_i$ de las macropartículas};
\node (fuentes) [calculo, below of=push] {Cálculo de las fuentes $\rho$, $\symbf{J}$ a partir de $\symbf{x}_i$, $\symbf{p}_i$};
\node (campos) [proceso, below of=fuentes, yshift=-0.5cm] {Actualización de los campos $\symbfcal{E}$, $\symbfcal{B}$ a partir de $\symbf{J}$};
\node (interp) [calculo, below of=campos, yshift=-0.5cm] {Interpolación de los campos $\symbfcal{E}$, $\symbfcal{B}$ a partir de $\symbf{J}$};
\node (preg) [decision, below of=interp, yshift=-0.5cm] {¿Converge?};
\node (fin) [fin, below of=preg] {Fin};

\draw [arrow] (inicio) -- (push);
\draw [arrow] (push) -- (fuentes);
\draw [arrow] (fuentes) -- (campos);
\draw [arrow] (campos) -- (interp);
\draw [arrow] (interp) -- (preg);
\draw [arrow] (preg) -- node[anchor=west] {Sí} (fin);
\draw [arrow] (preg) -- node[anchor=south,xshift=0.5cm] {No} +(-5,0) |- (push);

\end{tikzpicture}

\end{document}
