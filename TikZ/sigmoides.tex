% !TeX program = lualatex
% Exponential function divided in two different gaussian-like conitnuous functions
% Autor: Ismael Torres García
\documentclass[tikz,border=3pt]{standalone}
\usetikzlibrary{calc, arrows.meta}
\usepackage[math-style=ISO,bold-style=ISO]{unicode-math}
\setmainfont[Renderer=OpenType]{Minion Pro}          %% Fuente comercial MinionPro de Adobe para el texto
\setmathfont[                                        %% Fuente MinionMath de Johannes Küsner para las matemáticas
    Extension = .otf,
    Path = /Users/ytoga/Library/Fonts/,
    Scale = 1,
    Script = Math,
    SizeFeatures = {
        {Size = -6, Font = MinionMath-Tiny,
        Style = MathScriptScript},
        {Size = 6-8.4, Font = MinionMath-Capt,
        Style = MathScript},
        {Size = 8.4-, Font = MinionMath-Regular},
    },
]{MinionMath-Regular}
\setmathfont[                         
    Extension = .otf,
    Path = /Users/ytoga/Library/Fonts/,
    version = bold,
]{MinionMath-Bold}

\usepackage{pgfplots}
\pgfplotsset{compat=1.9}
\usepackage[dvipsnames]{xcolor}
\usepackage{siunitx}

%% FUNCIONES
\pgfmathdeclarefunction{sigmoide}{5}{%
  \pgfmathparse{#2 + #3/(1 + exp(0.001*#4*(#1-#5)))}%
}
\pgfmathdeclarefunction{fL}{3}{%
  \pgfmathparse{exp(-0.5*(max(#1,#2)/#3)^2)/exp(-0.5*(#2/#3)^2)}%
}
\pgfmathdeclarefunction{pgauss}{2}{%
  \pgfmathparse{exp(0.5*(#1/#2)^2)}%
}
\pgfmathdeclarefunction{ngauss}{2}{%
  \pgfmathparse{exp(-0.5*(#1/#2)^2)}%
}
%%%%%%%%%%%%%%%%%%%%%%%%%%
%%% COMIENZO DEL DOCUMENTO
%%%%%%%%%%%%%%%%%%%%%%%%%%
\begin{document}

\begin{tikzpicture}

%%%%%%%%%%%%%%%%
%%% DIBUJO 1 %%%
%%%%%%%%%%%%%%%%

%%% DEFINICIONES
%\def\rmin{15}           % en micras
%\def\rmax{25}           % en micras
%\def\L{\rmax-\rmin}     % en micras
%\def\ki{5000}           % en 1/metros
%\def\zi{4}            % en mm
%\def\zmax{5}            
%\def\ymax{1.1*sigmoide(0,\rmin,\L,\ki,\zi)}
%
%\begin{axis}[
%  every axis plot post/.append style={
%  mark=none, samples=1000, smooth},
%  enlargelimits=upper,
%  ymax=\ymax, ymin=0, 
%  xmax=1.1*\zmax, xmin=0, 
%  xlabel=$z(\unit{\mm})$, 
%  ylabel=$\sigma_{rL}(\unit{\um})$,
%  axis lines=middle,
%  axis line style={thick, -Latex[round]},
%  xtick={0, 1, 2, 3, 4, 5}, 
%  ytick={0, 0.2*\rmax, 0.4*\rmax, 0.6*\rmax, 0.8*\rmax, \rmax},
%  extra x ticks={0},
%  extra y ticks={0},
%  every axis x label/.style={at={(current axis.right of origin)},anchor=north west},
%  every axis y label/.style={at={(current axis.above origin)},anchor=south east}
%]
%
%%% REPRESENTACIÓN DE LA SIGMOIDE 
%\addplot[Tan, thick, domain=0:\rmax] {sigmoide(x,\rmin,\L,3000,\zi)};   
%\node[above right] at (axis cs:3.7,11.8) {$r_{ig,min}=\qty{15}{µm}$};

%%%%%%%%%%%%%%%%
%%% DIBUJO 2 %%%
%%%%%%%%%%%%%%%%

% DEFINICIONES
\def\rL{5}
\def\ru{18}
\def\sigL{15}
\def\sigu{20}
\def\rmax{60}
\def\ymax{1.1*fL(\rL,\rL,\sigL)}
\def\Aru{pgauss(\ru,\sigu)}

\begin{axis}[
  every axis plot post/.append style={
  mark=none, samples=100, smooth},
  enlargelimits=upper,
  clip=false,
  xmin=0, ymax=\ymax, 
  ymin=0, xmax=1.1*\rmax, 
  xlabel=$r(\unit{µm})$, 
  ylabel=$f(r)$,
  axis lines=middle,
  axis line style={thick, -Latex[round]},
  xtick={0, 10, 20, 30, 40, 50, 60}, 
  ytick={0, 0.2, 0.4, 0.6, 0.8, 1},
  extra x ticks={0},
  extra y ticks={0},
  every axis x label/.style={at={(current axis.right of origin)},anchor=north west},
  every axis y label/.style={at={(current axis.above origin)},anchor=south east}
]

% FUNCIÓN EXPONENCIAL INTERIOR 
\addplot[blue, thick, domain=0:\ru] {fL(x,\rL,\sigL)};   
\node[above right, blue] at (axis cs:8,0.9) {$f_L(r)$};

% FUNCIÓN EXPONENCIAL EXTERIOR 
\addplot[red, thick, domain=\ru:\rmax] 
{fL(\ru,\rL,\sigL)*\Aru*ngauss(x,\sigu)};
\node[above right, red] at (axis cs:20.4,0.45) {$f_u(r)$};

% TEXTO
\addplot[gray, densely dashed, thick] coordinates {(\ru,{fL(\ru,\rL,\sigL)}) (\ru,0)} node[anchor=south west] {\color{black}$r_u$}; 
\addplot[gray, densely dashed, thick] coordinates {(\rL,{fL(\rL,\rL,\sigL)}) (\rL,0)} node[anchor=south west] {\color{black}$r_L$};
\node[above right] at (axis cs:40,0.6) {$r_{u}=\qty{18}{µm}$};
\node[above right] at (axis cs:40,0.7) {$r_{L}=\qty{5}{µm}$};

\end{axis}
\end{tikzpicture}

\end{document}
